\documentclass[12pt]{article}

\input{glyphtounicode}
\pdfgentounicode=1

% Set page dimensions for comfortable reading
\setlength{\oddsidemargin}{1.65cm}
\setlength{\evensidemargin}{1.65cm}
\setlength{\textwidth}{14.4cm}
\setlength{\voffset}{-0.5cm}
\setlength{\topmargin}{0cm}
\setlength{\headheight}{0cm}
\setlength{\headsep}{0cm}
\setlength{\textheight}{23cm}
\setlength{\footskip}{1cm}

% Increase vertical space between lines
\renewcommand{\baselinestretch}{1.15}

% Disable paragraph indent
\setlength{\parindent}{0pt}

% Customize vertical space between paragraphs
%\setlength{\parskip}{.5\baselineskip}
%\setlength{\parskip}{1\baselineskip}

%\usepackage[T1]{fontenc}

% Use fontawesome5 icons (for home, phone, mail, LinkedIn and GitHub)
\usepackage{fontawesome5}

% Enable the combination of boldface and small caps 
\usepackage[notmath]{sansmathfonts}

% Enable clickable links
\usepackage{hyperref}

% Comment out one of the two lines below to enable/disable page numbering
%\pagestyle{empty} % disable page numbering
\pagestyle{plain}  % enable page numbering

% Define \mycolor (the color which will be used in this document)
\usepackage[dvipsnames]{xcolor}       % use this option for Fuchsia
%\usepackage[x11names]{xcolor}        % use this option for DodgerBlue4
\definecolor{myblue}{RGB}{76,117,180} % custom color example
\definecolor{mySlateBlue}{RGB}{97,112,140}
\definecolor{myFuchsia}{RGB}{77,0,77}
%\newcommand{\mycolor}{myFuchsia}
%\newcommand{\mycolor}{myblue}
%\newcommand{\mycolor}{Fuchsia}
%\newcommand{\mycolor}{DodgerBlue4}
\newcommand{\mycolor}{mySlateBlue}

\newlength\mylen
\setlength\mylen{4em}

% Customize the format of \section titles so that
% - a "section icon" appears instead of the section number;
% - section icons are placed within the left margin.
% (This is a hack! It kills section numbers!)
%
% Use \sectionicon{someIcon} to specify a section icon.
%
\usepackage[explicit]{titlesec}
\titlespacing{\section}{0pt}{*2.0}{*1.5}
\newcommand{\thesectionicon}{}
\newcommand{\sectionicon}[1]{\renewcommand{\thesectionicon}{#1}}
%\titleformat{\section}
%{\sffamily\bfseries\scshape\Large}{\llap{\makebox[1\mylen]{\textcolor{\mycolor}{\thesectionicon}}}}{-0.05em}{}
\titleformat{\section}
{\sffamily\bfseries\scshape\Large}{\llap{\makebox[1\mylen]{\textcolor{\mycolor}{\thesectionicon}}}}{-0.05em}{\colorbox{\mycolor}{\parbox[b]{\dimexpr\textwidth-0.05em-1\fboxsep-2\fboxrule\relax}{\vskip 0.25ex \textcolor{white}{#1} \vskip 0.25ex}}}

\begin{document}

\begin{minipage}[t]{0.25\textwidth}
  {\sffamily\bfseries\scshape\huge
    Marcio
    
         \vspace{1ex}
     
    Reverbel}
\end{minipage}
\hfill
\begin{minipage}[t]{0.65\textwidth}
  \begin{small}
  \begin{flushright}
  Woluwe-Saint-Lambert\ \ \textcolor{\mycolor}{\faHome\\}
  +32 (0)495 99 41 32\ \ \textcolor{\mycolor}{\faPhone\\}
  \texttt{mreverbel@protonmail.com}\ \ \textcolor{\mycolor}{\faEnvelope\\}
  % +32 495 99 41 32\ \ \faWhatsapp\\
  \href{https://linkedin.com/in/marcio-reverbel-223144208/}{\texttt{linkedin.com/in/marcio-reverbel-223144208/}}\ \ \textcolor{\mycolor}{\faLinkedin\\}
  \url{https://github.com/omledufromage}\ \ \textcolor{\mycolor}{\faGithub\\}
  \end{flushright}
  \end{small}
\end{minipage}
%\vspace{1ex}

\sectionicon{\faUser}
\section{In a Snapshot}

I am a student in the master's program in statistics at ULB, and also a pianist and music teacher. I see myself as someone who thrives in collaborative environments, although I am also capable of working independently whenever necessary. Diligent and curious, I keep a GitHub account with past university projects, as well as personal development projects. During my bachelor's at VUB I performed sentiment analysis (using the roBERTa model) on data scraped from Twitter, had a summer student job in data visualization, and was a student-assistant in Calculus and Linear Algebra courses for one year. I graduated from my bachelor's summa cum laude and currently maintain a 18.6 average GPA for my master's. 

\vspace{1.5ex}
My key interests are in the fields of mathematical statistics, data science, time series, design of experiments, macroeconomics and computer programming.

\sectionicon{\faGraduationCap}
\section{Education}

\begin{description}

\item[\textcolor{\mycolor}{MSc in Statistics}] | Université Libre de Bruxelles (ULB)

From 2023. Expected graduation: July, 2025.

\textbf{Relevant courses}: Graduate Statistics(19/20), Time Series (19/20), Mathematical Statistics II (18.5/ 20), Mathematical Statistics I (19/20), Multivariate and High-Dimensional Statistics (19/20), Regression Models (17/20), Méthodes de Sondage (17.5/20).
  
\textbf{Ongoing}: Algorithms for Big Data, Big Data: Distributed Data Management and Scalable Analytics, Statistical Foundations of Machine Learning, Inférence Robuste, Méthodes Non-Paramétriques, Analyse Statistique Multivariée, Topics in Probability Theory.

\item[\textcolor{\mycolor}{BSc in Business Economics}] | Vrije Universiteit Brussel (VUB)

Graduated in 2023. Specialization in Business and Technology.
  
GPA: 17.2/20 (with \textbf{greatest distinction -- \emph{summa cum laude}}).

Graduated from the faculty's excellence program, with extra credit in the form of student assistantships, extra courses taken (at ULB), and participation in the program committee meetings.

\textbf{Relevant courses}: Econometrics (20/20), Advanced Mathematics (20/20), Mathematics II (20/20), Mathematics I (20/20), Corporate Finance (20/20), Statistics II (19/20), Statistics I (19/20), Business Information Systems (17/20), Introduction to Microeconomics (17/20), Intermediate Microeconomics (17/20), Introduc\-tion to Macroeconomics (18/20), Intermediate Macroeconomics (16/20), IT Modeling and Economics (16/20).

Bachelor Paper: ``An analysis of the public perception of e-scooter regulations in Brussels (17/20)''.

\item[\textcolor{\mycolor}{Data Science Specialization}] | Coursera -- John Hopkins University
  
%Completed courses: The Data Scientist’s Toolbox (KEQJ386S4RRV), R Programming (Q53GBDJHN7FJ), Getting and Cleaning Data (LQJXCFVQ6ES6).
  
Completed courses: The Data Scientist’s Toolbox, R Programming, Getting and Cleaning Data.  

\item[\textcolor{\mycolor}{MA in Music, Piano}] | Koninklijk Conservatorium Brussel

Graduated in 2019.

\item[\textcolor{\mycolor}{BA in Music, Piano}] | University of São Paulo

Graduated in 2014.

Awarded a one-year FAPESP Research Fellowship for my project on the formal analysis of Debussy and Messiaen Préludes.

\textbf{Relevant courses}: Music Analysis I (10/10), Music Analysis II (10/10).

\end{description}

\sectionicon{\faToolbox}
\section{Selected Work Experience}
\begin{description} 
\item[\textcolor{\mycolor}{Student Job}] | Vrije Universiteit Brussel

One-month research project, supervised by Prof. Vincent Ginis, exploring data visualization of music. I analyzed the formal structure of pieces of music, based on Martin Wattenberg’s “The Shape of Song” project and its implementation of arc diagrams using suffix trees.
  
From July to August, 2022.

\item[\textcolor{\mycolor}{Mathematics Student-Assistant}] | Vrije Universiteit Brussel

As part of the Excellence program in my Bachelor’s, I was tasked with helping the Mathematics professor and assistants throughout the year, specifically during the weekly exercise sessions for the Mathematics I and Mathematics II courses from the Business Economics bachelor program.

From September 2021 to June 2022.

\item[\textcolor{\mycolor}{Piano Teacher}] | BRIMA - Brussels International Music Academy\footnote{The school changed name in 2022. It was previously known as the École de Musique Tchaikovsky.}
  
From 2020 until 2023.

\item[\textcolor{\mycolor}{Piano Teacher}] | Cap sur la Musique

From 2018 until 2021

\end{description}

% hack to push the footnote to the bottom of the page
%\vspace{\stretch{1}}
%\mbox{\ }

\sectionicon{\faTools}
\section{Skills}

\begin{minipage}[t]{0.6\textwidth}
\begin{itemize}
  \setlength{\itemsep}{0pt}
  \item Programming Languages:
  \begin{itemize}
    \setlength{\itemsep}{0pt}
  \item \textbf{Python}
      \begin{tabular}[t]{l}
          (numpy, pandas, matplotlib, \\
           statsmodels, scipy, scikit-learn)\\
    \end{tabular}
    \item \textbf{R} (ggplot2, dplyr)
    \item \textbf{Mathematica}
    \item \textbf{C}
      \begin{tabular}[t]{l}
          (I wrote programs in C with\\
          approximately 300 lines of code.)\\
    \end{tabular}
  \end{itemize}
\end{itemize}
\end{minipage}
\hspace{\stretch{1}}
\begin{minipage}[t]{0.35\textwidth}
\begin{itemize}
  \setlength{\itemsep}{0pt}
  \item Git
  \item Microsoft:\\
    \begin{tabular}[t]{l}
      Access, Excel, \\
      Power BI,\\
      Project, Visio \\
    \end{tabular}
  \item \LaTeX
\end{itemize}
\hfill
\end{minipage}

\sectionicon{\faLanguage}
\section{Languages}

\begin{minipage}[t]{0.45\textwidth}
\begin{itemize}
\setlength{\itemsep}{0pt}
\item Portuguese (Fluent – Native)
\item English (Fluent – Bilingual)
\item French (Professional – B2)
\end{itemize}
\end{minipage}
\begin{minipage}[t]{0.55\textwidth}
\begin{itemize}
\setlength{\itemsep}{0pt}
\item German (Beginner – A2)
\item Chinese Mandarin (Beginner – HSK 2)
\end{itemize}
\hfill
\end{minipage}

\sectionicon{\faRunning}
\section{Activities}

During the year of 2011, I did volunteer work for Crea+, an NGO dedicated to giving reinforcement mathematics lessons to children from public schools in São Paulo, Brazil.

\smallskip
\textbf{Other Interests}: Board games, ping-pong, swimming, jazz, early music, biking, Chinese and Japanese cuisine.

\end{document}
